% !TeX spellcheck = en_GB
\documentclass[handout]{beamer}\mode<presentation>{\usetheme{AMSCesenaPurpleAndGold}}
% \documentclass[presentation]{beamer}\mode<presentation>{\usetheme{AMSCesenaBleu}}
%%%%

\usepackage{common}
\usepackage{courses}

\title[L0 -- About the Lab]{About the Lab}
%
\subtitle[SD]
{Distributed Systems / Technologies\\\scriptsize Sistemi Distribuiti / Tecnologie}
%
\author[Ciatto \and Omicini]
{\emph{Giovanni Ciatto} \and Andrea Omicini\\
\texttt{giovanni.ciatto@unibo.it \and andrea.omicini@unibo.it}}
%
\institute[DISI, Univ. Bologna]
{Dipartimento di Informatica -- Scienza e Ingegneria (DISI)\\\textsc{Alma Mater Studiorum} -- Universit{\`a} di Bologna a Cesena}
%
\date[A.Y. 2020/2021]{Academic Year 2020/2021}

\setbeamercovered{transparent}

\AtBeginSection[]
{
	\begin{frame}[c]\frametitle{Next in Line\ldots}
%		\begin{multicols}{2}
			\tableofcontents[sectionstyle=show/shaded, subsectionstyle=show/hide, subsubsectionstyle=hide/hide]
%		\end{multicols}
	\end{frame}
}

\begin{document}

\maketitle

\begin{frame}[c]\frametitle{Outline}
    % \begin{multicols}{2}
	    \tableofcontents[sectionstyle=show/show, subsectionstyle=show/show, subsubsectionstyle=show/show]
    % \end{multicols}
\end{frame}

\section{Vision}

\subsection{Dimensions of Software (Engineering)}

\begin{frame}[allowframebreaks]
\frametitle{Dimensions of Software (Engineering)}

    \begin{block}{In your Bachelor's you learnt how to properly \textbf{design}}
    	\begin{itemize}
    		\item the \alert{structure} of \emph{non-concurrent} systems
    		\item the \alert{behaviour} of most common algorithm in computer science (CS)
    		\item and how to implement \emph{non-distributed} systems
    	\end{itemize}
    \end{block}

	\framebreak
	
	\begin{alertblock}{However, there are 3 major \textbf{dimensions} of Software (Engineering)}
		\begin{description}
			\item[structure] -- how domain entities are composed and relate
			\item[behaviour] -- which algorithms govern their behaviour
			\item[\textbf{interaction}] -- when they are allowed to exchange information, and how
		\end{description}
	\end{alertblock}
	%
	\hint{these are just insights, not definitions}
	
	\framebreak
	
	\begin{exampleblock}{In concurrent or even \textbf{distributed} systems}
		\begin{itemize}
			\item the interaction dimension is of paramount importance
			\item as well as aspects related to 
			%
			\begin{itemize}
				\item the \alert{software architecture}
				\item the \alert{control flow(s)}
			\end{itemize}
		\end{itemize}
	\end{exampleblock}

	\bigskip

%	\begin{exampleblock}{In fact}
	One can describe and design concurrent/distributed systems as:
	%
	\begin{itemize}
		\item systems composed by several \alert{pro-active} entities
		%
		\begin{itemize}
			\item[ie] entities \alert{encapsulating} ($\approx$ having) their own \alert{control flow}
		\end{itemize}
		\item \alert{interacting} among each others
		\item possibly, over the \alert{network}
	\end{itemize}
%	\end{exampleblock}

\end{frame}

\subsection{Goal of the Lab}

\begin{frame}
	\frametitle{Goal of the Lab}
	
	During the Lab lessons, we will progressively introduce you
	%
	\vfill
	%
	\begin{itemize}
		\item to the \alert{design} of concurrent \& distributed systems (C\&DS),
		
		\vfill
		
		\item while providing a taste of 
		%
		\begin{itemize}
			\item the challenges \& subtleties posed by C\&DS
			\item the most common architectural choices for C\&DS
			\item the most adequate design \& programming abstractions for C\&DS
		\end{itemize}
	
	\end{itemize}
	
\end{frame}

\subsection{Engineering concurrent \& distributed systems}

\begin{frame}[allowframebreaks]
	\frametitle{Engineering concurrent \& distributed systems (C\&DS)}
	
	Engineering C\&DS requires encompassing several, progressive phases
	%
	\begin{enumerate}
		\item modelling
		\item design
		\item implementation
		\item deployment
	\end{enumerate}
	
	\framebreak
	
	\begin{block}{\textbf{Modellers} must define \textbf{domain}-related aspects}
		\begin{enumerate}
			\item which and how many sorts \alert{domain entities} are involved into a system,
			\item whether they are \alert{pro-active} or \alert{reactive},
			\item what's their \alert{behavioural specification}, and
			\item when with whom they can \alert{interact}
		\end{enumerate}
	\end{block}

	\bigskip
	
	\begin{itemize}
		\item Modellers must thus be able to govern:
		%
		\begin{itemize}
			\item systems composed by several \alert{loci of control}, 
			\item as well as the \alert{communication means} among them
		\end{itemize}
	
		\bigskip
	
		\item[$\rightarrow$] OOP patterns and UML are not adequate to capture these aspects
		%
		\begin{itemize}
			\item we need higher level notions such as \alert{agents}, \alert{tuple spaces}, etc
		\end{itemize}
	\end{itemize}

	\framebreak
	
	\begin{block}{\textbf{Designers} must address \textbf{architectural} aspects}
		\begin{itemize}
			\item whether the interaction involves \alert{two or more} entities
			\item which entities must \alert{initiate} interaction (acting as clients)
			\item which entities must \alert{wait} for interaction to be initiated (servers)
			\item whether interaction should be me \alert{mediated} or not
			\item which information should be provided by each entity, and \alert{when}
			\item how should this information be \alert{structured}
		\end{itemize}
	\end{block}

	\bigskip
	
	\begin{itemize}
		\item Designers must thus be able to govern:
		%
		\begin{itemize}
			\item interaction patterns (request-response, publish-subscribe, etc.),
			\item and their properties 
		\end{itemize}
		
		\bigskip
		
		\item[$\rightarrow$] method calls or sockets from OOP programming are not adequate
		%
		\begin{itemize}
			\item we need higher level notions such as \alert{FIPA's IP}, or \alert{\linda{}'s primitives}
		\end{itemize}
	\end{itemize}

	\framebreak
	
	\begin{block}{\textbf{Engineers} must tackle \textbf{technical} aspects}
		\begin{enumerate}
			\item let several loci of control \alert{co-exist} in an efficient \& scalable way\ldots
			%
			\begin{itemize}
				\item possibly in spite of \alert{distribution}
				\item possibly in spite of different \alert{execution platforms}
			\end{itemize}
			\item make their interaction \alert{observable \& debuggable}
			\item chose the most adequate technological protocol\ldots 
			\item \ldots and data representation format 
			%
			\begin{itemize}
				\item in order to keep the system open, interoperable, evolvable, yet efficient
			\end{itemize}
		\end{enumerate}
	\end{block}

	\medskip

	\begin{itemize}
		\item Engineers must thus be able to govern:
		%
		\begin{itemize}
			\item middleware technologies (e.g. \jade{})
			\item architectural styles (e.g. ReST)
			\item application-level protocols (e.g. HTTP),
			\item data-representation formats (e.g. JSON, YAML)
		\end{itemize}

	\end{itemize}
	
\end{frame}

\section{In practice}

\subsection{Structure of the Lab}

\begin{frame}[allowframebreaks]
	\frametitle{Structure of the Lab}
	
	An end-to-end path providing a taste of all the aforementioned issues (showing how main notions are built \& exploited C\&DS)
	%
	\medskip
	%
	\begin{enumerate}
		\item you'll firstly learn how to program non-distributed concurrent systems
		%
		\begin{itemize}
			\item through \alert{asynchrnous programming}
		\end{itemize}
	
		\medskip
	
		\item you'll then practice with systems involving several loci of control
		%
		\begin{itemize}
			\item e.g. \alert{multi-agent systems}
			\item learning how \jade{} agents work and can be built from scratch
		\end{itemize}
	
		\medskip
		
		\item you'll then control agent's interactions through \alert{\linda{}'s tuple spaces}
		%
		\begin{itemize}
			\item learning how they can be realised from scratch
			\item or exploited in flexible ways to implement several \alert{interaction patterns}
		\end{itemize}
		
		\framebreak
		
		\item you'll then leverage \alert{web services} to make tuple spaces distributed
		%
		\begin{itemize}
			\item learning about web-services and the ReST architectural style
			\item learning how non-distributed systems can be made distributed
		\end{itemize}
		
		\medskip
		
		\item you'll then exploit tuple spaces to let agents interact \alert{over the network}
		%
		\begin{itemize}
			\item learning how to make distribution transparent in your system
			\item learning about the client- and server-side of a distributed system
		\end{itemize}
	
		\medskip
	
		\item you should finally have built your own framework for C\&DS
		%
		\begin{itemize}
			\item enabling experiments and exercises about, e.g.:
			%
			\begin{itemize}
				\item deployment
				\item replication
				\item consistency
			\end{itemize}
		\end{itemize}
		
	\end{enumerate}
	
\end{frame}


\maketitle

%%%%%%%%%%%%%%%%%%%%%%%%%%%%%%%%%%%%%%%%%%%%%%%%%%%%%%%%%%%%%%%%%%%%%%%%%%%%%%%
\end{document}
%%%%%%%%%%%%%%%%%%%%%%%%%%%%%%%%%%%%%%%%%%%%%%%%%%%%%%%%%%%%%%%%%%%%%%%%%%%%%%%%
