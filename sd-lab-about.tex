% !TeX spellcheck = en_GB
\documentclass[presentation]{beamer}\mode<presentation>{\usetheme{AMSBolognaFC}}
\setbeamertemplate{bibliography item}{\insertbiblabel}

\usepackage{common}

\title[L0 -- About the Lab]{About the Lab}
%
\subtitle[SD]
{Distributed Systems\\\scriptsize Technologies}
%
\author[Ciatto \and Omicini]
{\emph{Giovanni Ciatto} \and Andrea Omicini\\
\texttt{giovanni.ciatto@unibo.it \and andrea.omicini@unibo.it}}
%
\institute[DISI, Univ. Bologna]
{Dipartimento di Informatica -- Scienza e Ingegneria (DISI)\\\textsc{Alma Mater Studiorum} -- Universit{\`a} di Bologna a Cesena}
%
\date[A.Y. 2020/2021]{Academic Year 2020/2021}

\setbeamercovered{transparent}

\AtBeginSection[]
{
\begin{frame}[c,noframenumbering]\frametitle{Next in Line\ldots}
%	\begin{multicols}{2}
	\tableofcontents[sectionstyle=show/shaded, subsectionstyle=show/hide, subsubsectionstyle=hide/hide]
%	\end{multicols}
\end{frame}
}

\AtBeginSubsection[]
{
    \begin{frame}[c,noframenumbering]\frametitle{Next in Line\ldots}
        %	\begin{multicols}{2}
        \tableofcontents[sectionstyle=show/shaded, subsectionstyle=show/hide, subsubsectionstyle=hide/hide]
        %	\end{multicols}
    \end{frame}
}

\begin{document}

\maketitle

\begin{frame}[c]\frametitle{Outline}
    % \begin{multicols}{2}
	    \tableofcontents[sectionstyle=show/show, subsectionstyle=show/show, subsubsectionstyle=show/show]
    % \end{multicols}
\end{frame}

\section{Vision}

\subsection{Dimensions of Software (Engineering)}

\begin{frame}[c,allowframebreaks]{Dimensions of Software (Engineering)}

    \begin{block}{In your Bachelor's you learnt how to properly \textbf{design}}
    	\begin{itemize}
    		\item the \alert{structure} of \emph{non-concurrent} systems
    		\item the \alert{behaviour} of most common algorithm in computer science (CS)
    		\item and how to implement \emph{non-distributed} systems
    	\end{itemize}
    \end{block}

	\framebreak

	\begin{alertblock}{However, there are 3 major \textbf{dimensions} of Software (Engineering)}
		\begin{description}
			\item[structure] -- how domain entities are composed and relate
			\item[behaviour] -- which algorithms govern their behaviour
			\item[\textbf{interaction}] -- when they are allowed to exchange information, and how
		\end{description}
	\end{alertblock}
	%
	\hint{these are just insights, not definitions}

	\framebreak

	\begin{exampleblock}{In concurrent and distributed systems}
		\begin{itemize}
			\item the interaction dimension is of paramount importance
			\item as well as aspects related to
			%
			\begin{itemize}
				\item the \alert{software architecture}
				\item the \alert{control flow(s)}
			\end{itemize}
		\end{itemize}
	\end{exampleblock}

	\bigskip

%	\begin{exampleblock}{In fact}
	One can describe and design concurrent/distributed systems as:
	%
	\begin{itemize}
		\item systems composed by several \alert{pro-active} entities
		%
		\begin{itemize}
			\item[ie] entities \alert{encapsulating} ($\approx$ having) their own \alert{control flow}
		\end{itemize}
		\item \alert{interacting} among each others
		\item possibly, over the \alert{network}
	\end{itemize}
%	\end{exampleblock}

\end{frame}

\subsection{Goal of the Lab}

\begin{frame}[c]{Goal of the Lab}

	During the Lab lessons, we will progressively introduce you
	%
	\vfill
	%
	\begin{itemize}
		\item to the \alert{design} of concurrent \& distributed systems (C\&DS)

		\vfill

		\item while providing a taste of
		%
		\begin{itemize}
			\item the challenges \& subtleties posed by C\&DS
			\item the most common architectural choices for C\&DS
			\item the most adequate design \& programming abstractions for C\&DS
		\end{itemize}

	\end{itemize}

\end{frame}

\subsection{Engineering Concurrent \& Distributed Systems}

\begin{frame}[c,allowframebreaks]{Engineering Concurrent \& Distributed Systems (C\&DS)}

	Engineering C\&DS requires encompassing several, progressive phases
	%
	\begin{enumerate}
		\item modelling
		\item design
		\item implementation
		\item deployment
	\end{enumerate}

	\framebreak

	\begin{block}{\textbf{Modellers} must define \textbf{domain}-related aspects}
		\begin{enumerate}
			\item which and how many sorts \alert{domain entities} are involved into a system,
			\item whether they are \alert{pro-active} or \alert{reactive},
			\item what's their \alert{behavioural specification}, and
			\item when and with whom they can \alert{interact}
		\end{enumerate}
	\end{block}

	\bigskip

	\begin{itemize}
		\item modellers must thus be able to govern
		%
		\begin{itemize}
			\item systems composed by several \alert{loci of control},
			%
			\begin{itemize}
				\item[eg] threads, event-loops, actors, or \alert{agents}
			\end{itemize}

			\item as well as the \alert{communication means} among them
			%
			\begin{itemize}
				\item[eg] channels, streams, messages, \alert{tuple spaces}
			\end{itemize}
		\end{itemize}

		\bigskip

		\item[$\rightarrow$] OOP patterns and UML are not adequate to capture these aspects
		%
		\begin{itemize}
			\item we need higher level notions such as \alert{agents}, \alert{tuple spaces}, etc
		\end{itemize}
	\end{itemize}

	\framebreak

	\begin{block}{\textbf{Designers} must address \textbf{architectural} aspects}
		\begin{itemize}
			\item whether the interaction involves \alert{two or more} entities
			\item which entities must \alert{initiate} interaction (acting as clients)
			\item which entities must \alert{wait} for interaction to be initiated (servers)
			\item whether interaction should be \alert{mediated} or not
			\item which information should be provided by each entity, and \alert{when}
			\item how should this information be \alert{structured}
		\end{itemize}
	\end{block}

	\bigskip

	\begin{itemize}
		\item designers must thus be able to govern
		%
		\begin{itemize}
			\item interaction patterns (request-response, publish-subscribe, etc.),
			\item and their properties
		\end{itemize}

		\bigskip

		\item[$\rightarrow$] method calls or sockets from OOP programming are not adequate
		%
		\begin{itemize}
			\item we need higher level notions such as \alert{FIPA's IP}, or \alert{\linda{}'s primitives}
		\end{itemize}
	\end{itemize}

	\framebreak

	\begin{block}{\textbf{Engineers} must tackle \textbf{technical} aspects}
		\begin{enumerate}
			\item let several loci of control \alert{co-exist} in an efficient \& scalable way\ldots
			%
			\begin{itemize}
				\item possibly in spite of \alert{distribution}
				\item possibly in spite of different \alert{execution platforms}
			\end{itemize}
			\item make their interaction \alert{observable \& debuggable}
			\item chose the most adequate technological protocol\ldots
			\item \ldots and data representation format
			%
			\begin{itemize}
				\item in order to keep the system open, interoperable, evolvable, yet efficient
			\end{itemize}
		\end{enumerate}
	\end{block}

	\medskip

	\begin{itemize}
		\item engineers must thus be able to govern
		%
		\begin{itemize}
			\item middleware technologies (e.g. \jade{})
			\item architectural styles (e.g. ReST)
			\item application-level protocols (e.g. HTTP),
			\item data-representation formats (e.g. JSON, YAML)
		\end{itemize}

	\end{itemize}

\end{frame}

\section{Organisation}

\subsection{Structure of Lab lectures}

\begin{frame}[c,allowframebreaks]{Structure of the Lab}

	An end-to-end path providing a taste of all the aforementioned issues (showing how main notions are built \& exploited C\&DS)
	%
	\medskip
	%
	\begin{enumerate}
		\item you will firstly learn how to program non-distributed concurrent systems
		%
		\begin{itemize}
			\item through \alert{asynchronous programming}
		\end{itemize}

		\medskip

		\item you will then practice with systems involving several loci of control
		%
		\begin{itemize}
			\item e.g. \alert{multi-agent systems}
			\item learning how \jade{} agents work and can be built from scratch
		\end{itemize}

		\medskip

		\item you will then control agent's interactions through \alert{\linda{}'s tuple spaces}
		%
		\begin{itemize}
			\item learning how they can be realised from scratch
			\item or exploited in flexible ways to implement several \alert{interaction patterns}
		\end{itemize}

		\framebreak

		\item you will then leverage \alert{web services} to make tuple spaces distributed
		%
		\begin{itemize}
			\item learning about web-services and the ReST architectural style
			\item learning how non-distributed systems can be made distributed
		\end{itemize}

		\medskip

		\item you will then exploit tuple spaces to let agents interact \alert{over the network}
		%
		\begin{itemize}
			\item learning how to make distribution transparent in your system
			\item learning about the client- and server-side of a distributed system
		\end{itemize}

		\medskip

		\item you should finally have built your own framework for C\&DS
		%
		\begin{itemize}
			\item enabling experiments and exercises about, e.g.
			%
			\begin{itemize}
				\item deployment
				\item replication
				\item consistency
			\end{itemize}
		\end{itemize}

	\end{enumerate}

\end{frame}

\subsection{Required Technologies and Skills}

\begin{frame}[c,allowframebreaks]{Required Technologies and Skills}

	\begin{block}{Legend}
		\begin{multicols}{2}
			\begin{itemize}
				\item[$\checkmark$] we assume you know this topic
				\item[$\rightarrow$] we teach this topic
			\end{itemize}
		\end{multicols}
	\end{block}

	\begin{alertblock}{Required}
		\begin{itemize}
			\item[$\checkmark$] distributed version control systems (DVCS) \& \alert{Git}
			%
			\begin{itemize}
				\item useful resources: \ccite{pianiniDvcs, proGit}
			\end{itemize}

			\vfill

			\item[$\rightarrow$] build automation tools and \alert{Gradle}
			%
			\begin{itemize}
				\item useful resources: \ccite{pianiniBuildAutomation, gradleUserGuide}
			\end{itemize}

			\vfill

			\item Java programming, and, in particular:
			%
			\begin{itemize}
				\item[$\checkmark$] collections API (useful resources: \ccite{Naftalin2006, Bloch2008, JavaCollectionsCheatsheets})
				\item[$\checkmark$] streams and lambdas API (useful resources: \ccite{Warburton2014, Bloch2008})
				\item[$\checkmark$] multi-threading API (useful resources: \ccite{Lea1999, Oaks2004, Garg2004, Goetz2006})
				\item[$\rightarrow$] asynchronous programming API
			\end{itemize}

		\end{itemize}
	\end{alertblock}

	\begin{exampleblock}{Useful}
		\begin{itemize}
			\item[$\checkmark$] shell scripting and \alert{Bash}
			\item[$\rightarrow$] containerisation and \alert{Docker}
			\item[$\checkmark$] basic IDE configuration and usage (\alert{Eclipse} or \alert{IntelliJ Idea})
		\end{itemize}
	\end{exampleblock}

	\begin{alertblock}{Setting up your own environment}
		\begin{itemize}
			\item[!] follow the instructions provided here: \ccite{envSetup}
		\end{itemize}
	\end{alertblock}

\end{frame}

\section{Conventions and Suggestions}

\begin{frame}[c]{About Lab Exercises}

	\begin{itemize}
		\item you are supposed to learn something, not just solving puzzles
		%
		\begin{itemize}
			\item no deadline, no mark
			\item[$\rightarrow$] take your time
			\item[$\rightarrow$] ask for help on the forum
			\item[$\rightarrow$] compare/discuss with your colleagues
		\end{itemize}

		\vfill

		\item comprehension of Lab-related topics is checked before final exam:
		%
		\begin{itemize}
			\item \alert{expected workflow:} Lab activity check $\rightarrow$ Admission to final exam
			\item check performed by tutor(s)
			\item students must explicitly ask for the check, via email
		\end{itemize}

		\vfill

		\item exercises are provided as GitLab repositories hosting Gradle projects
		%
		\begin{itemize}
			\item you need a GitLab account $\rightarrow$ follow the instructions described in \ccite{envSetup}
		\end{itemize}

        \vfill

        \item slides are provided as \emph{versioned} PDF, through GitHub releases
        %
        \begin{itemize}
            \item[eg] \uurl{https://github.com/gciatto-unibo/sd-lab-environment-configuration/releases}
            \item[eg] \uurl{https://github.com/gciatto-unibo/sd-lab-about/releases}
        \end{itemize}

	\end{itemize}

\end{frame}

\begin{frame}[c]{About Exercises Submissions}

    \begin{itemize}
        \item solutions must be pushed on the same GitLab repository exercises have been cloned from, \alert{through Git}

        \vfill

        \item to this end, you create a GitLab account, following instructions provided in \ccite{envSetup}:
        %
        \begin{itemize}
            \item possibly, using your institutional credentials \texttt{\alert{name.surnameN}@studio.unibo.it}
            \item possibly, using \texttt{\alert{name.surnameN}} as username
        \end{itemize}

        \vfill

        \item after that, you are supposed to request access on the DS 20-21 GitLab group:
        %
        \begin{itemize}
            \item \url{https://gitlab.com/pika-lab/courses/ds/ay2021}
        \end{itemize}

        \vfill

        \item solutions must be pushed on a branch named \alert{\texttt{submissions/\textit{name.surnameN}}}, as described in \ccite{envSetup}
        %
        \begin{itemize}
            \item one solution per student ($\implies$ no group submissions)
            \item solutions provided elsewhere will be ignored
        \end{itemize}
    \end{itemize}
\end{frame}

\begin{frame}[c]{About Fora}
    \begin{itemize}
        \item use the forum as much as possible\\
        \uurl{https://virtuale.unibo.it/mod/forum/view.php?id=331530}
        %
        \begin{itemize}
            \item don't be shy :)
            \item ask for help if you need it
            \item ask why if you are curious
            \item compare you solutions
            \item be critical and provide suggestions if you feel so
        \end{itemize}
    \end{itemize}
\end{frame}

\section{The Final Project}

\begin{frame}[c,allowframebreaks]{About Projects}
    \begin{itemize}
        \item detailed rules here\\
        \uurl{https://virtuale.unibo.it/mod/page/view.php?id=331533}

        \vfill

        \item workflow overview
        %
        \begin{enumerate}
            \item choose a project or propose one
            \item reserve your project on the \href{https://virtuale.unibo.it/mod/forum/view.php?id=331532}{Projects forum}
            \item submit an \alert{initial report}, describing your own requirements
            \item receive a Git repository for tracking the development of your artefacts
            \item develop your projects
            \item write the \alert{final report}
            \item submit your project \alert{code \& report} and ask for lab activity check
            \item set up an appointment for discussing your project
        \end{enumerate}

        \vfill

        \item group projects are allowed (max 4 persons)
        %
        \begin{itemize}
            \item rule of thumb: $\sim90$ working hours per person per project
        \end{itemize}
    \end{itemize}
\end{frame}

%===============================================================================
\section*{}
%===============================================================================
\frame{\titlepage}

%===============================================================================
\section*{\bibname}
%===============================================================================

\setbeamertemplate{page number in head/foot}{}
%\\\\\\\\\\\\\\\\\\\\\
\begin{frame}[t,allowframebreaks,noframenumbering]\frametitle{\refname}
	% \begin{frame}[c]\frametitle{\refname}
	%	\footnotesize
	%	\scriptsize
		\tiny
	\bibliographystyle{plain}
	\bibliography{sd-lab-about}
\end{frame}
%\\\\\\\\\\\\\\\\\\\\\

%%%%%%%%%%%%%%%%%%%%%%%%%%%%%%%%%%%%%%%%%%%%%%%%%%%%%%%%%%%%%%%%%%%%%%%%%%%%%%%
\end{document}
%%%%%%%%%%%%%%%%%%%%%%%%%%%%%%%%%%%%%%%%%%%%%%%%%%%%%%%%%%%%%%%%%%%%%%%%%%%%%%%%
