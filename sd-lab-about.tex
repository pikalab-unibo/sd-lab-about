%%%%%%%%%%%%%%%%%%%%%%%%%%%%%%%%%%%%%%%%%%%%%%%%%%%%%%%%%%%%%%%%%%%%%%%%%%%%%%%%
% SD Lab -- About the course
% Giovanni Ciatto
% Alma Mater Studiorum - Università di Bologna
% mailto:giovanni.ciatto@unibo.it
%%%%%%%%%%%%%%%%%%%%%%%%%%%%%%%%%%%%%%%%%%%%%%%%%%%%%%%%%%%%%%%%%%%%%%%%%%%%%%%%
%\documentclass[handout]{beamer}\mode<handout>{\usetheme{default}}
%
\documentclass[presentation]{beamer}\mode<presentation>{\usetheme{AMSBolognaFC}}
%\documentclass[handout]{beamer}\mode<handout>{\usetheme{AMSBolognaFC}}
%%%%%%%%%%%%%%%%%%%%%%%%%%%%%%%%%%%%%%%%%%%%%%%%%%%%%%%%%%%%%%%%%%%%%%%%%%%%%%%%
\usepackage{sd-lab-common}
\usepackage{sd-lab-about}
%%%%%%%%%%%%%%%%%%%%%%%%%%%%%%%%%%%%%%%%%%%%%%%%%%%%%%%%%%%%%%%%%%%%%%%%%%%%%%%%
\title[\currentLab{} -- About]{
	About the course
}
%
\subtitle{\courseName{} (\courseAcronym) / Module \moduleN{}}
%
\author[\sspeaker{\gcShort} \& \mmShort]{
	\speaker{\gcFull} \and \mmFull
	\\
	\gcEmail \and \mmEmail
}
%
\institute[\disiShort, \uniboShort]{\disi{} (\disiShort)\\\unibo}
%
\date[A.Y. \academicYear{}]{Academic Year \academicYear{}}
%
\begin{document}

\maketitle

\begin{frame}[c]\frametitle{Outline}
    % \begin{multicols}{2}
        \tableofcontents[sectionstyle=show/show, subsectionstyle=show/show, subsubsectionstyle=show/show]
    % \end{multicols}
\end{frame}

\section{Vision}

\subsection{Dimensions of Software (Engineering)}

\begin{frame}[c,allowframebreaks]{Dimensions of Software (Engineering)}

    \begin{block}{In your Bachelor's you learnt how to properly \textbf{design}}
        \begin{itemize}
            \item the \alert{structure} of \emph{non-concurrent} systems
            \item the \alert{behaviour} of most common algorithm in computer science (CS)
            \item and how to implement \emph{non-distributed} systems
        \end{itemize}
    \end{block}

    \framebreak

    \begin{alertblock}{However, there are 3 major \textbf{dimensions} of Software (Engineering)}
        \begin{description}
            \item[structure] -- how domain entities are composed and related
            \item[behaviour] -- which algorithms govern their behaviour
            \item[\textbf{interaction}] -- when they are allowed to exchange information, and how
        \end{description}
    \end{alertblock}
    %
    \hint{these are just insights, not definitions}

    \framebreak

    \begin{exampleblock}{In concurrent and distributed systems}
        \begin{itemize}
            \item the interaction dimension is of paramount importance
            \item as well as aspects related to
            %
            \begin{itemize}
                \item the \alert{software architecture}
                \item the \alert{control flow(s)}
            \end{itemize}
        \end{itemize}
    \end{exampleblock}

    \bigskip

%    \begin{exampleblock}{In fact}
    One can describe and design concurrent/distributed systems as:
    %
    \begin{itemize}
        \item systems composed by several \alert{pro-active} entities
        %
        \begin{itemize}
            \item[ie] entities \alert{encapsulating} ($\approx$ having) their own \alert{control flow}
        \end{itemize}
        \item \alert{interacting} among each others
        \item possibly, over the \alert{network}
    \end{itemize}
%    \end{exampleblock}

\end{frame}

\subsection{Goal of the Lab}

\begin{frame}[c]{Goal of the Lab}

    Within Lab lectures, we will progressively introduce you
    %
    \vfill
    %
    \begin{itemize}
        \item the fundamental aspects of concurrent \& distributed systems (C\&DS)

        \vfill

        \item while providing a taste of
        %
        \begin{itemize}
            \item the challenges \& subtleties posed by C\&DS
            \item the most adequate design \& programming abstractions for C\&DS
            %
            \begin{itemize}
                \item and some technologies supporting them
            \end{itemize}
        \end{itemize}

    \end{itemize}

\end{frame}

\subsection{Engineering Concurrent \& Distributed Systems}

\begin{frame}[c,allowframebreaks]{Engineering Concurrent \& Distributed Systems (C\&DS)}

    Engineering C\&DS requires encompassing several, progressive phases
    %
    \begin{enumerate}
        \item modelling
        \item design
        \item implementation
        \item deployment
    \end{enumerate}

    \framebreak

    \begin{block}{\textbf{Modellers} must define \textbf{domain}-related aspects}
        \begin{enumerate}
            \item which and how many sorts \alert{domain entities} are involved into a system,
            \item whether they are \alert{pro-active} or \alert{reactive},
            \item what's their \alert{behavioural specification}, and
            \item when and with whom they can \alert{interact}
        \end{enumerate}
    \end{block}

    \bigskip

    \begin{itemize}
        \item modellers must thus be able to govern
        %
        \begin{itemize}
            \item systems composed by several \alert{loci of control},
            %
            \begin{itemize}
                \item[eg] threads, event-loops, actors, or \alert{agents}
            \end{itemize}

            \item as well as the \alert{communication primitives} among them
            %
            \begin{itemize}
                \item[eg] channels, streams, messages, or tuple spaces/centres
            \end{itemize}
        \end{itemize}

        \bigskip

        \item[$\rightarrow$] OOP abstractions are not adequate to capture these aspects
        %
        \begin{itemize}
            \item we need higher level abstractions such as agents, services, etc.
        \end{itemize}
    \end{itemize}

    \framebreak

    \begin{block}{\textbf{Designers} must address \textbf{architectural} aspects}
        \begin{itemize}
            \item whether the interaction involves \alert{two or more} entities
            \item which entities must \alert{initiate} interaction (acting as clients)
            \item which entities must \alert{wait} for interaction to be initiated (servers)
            \item whether interaction should be \alert{mediated} or not
            \item which information should be provided by each entity, and \alert{when}
            \item how should this information be \alert{structured}
        \end{itemize}
    \end{block}

    \bigskip

    \begin{itemize}
        \item designers must thus be able to govern
        %
        \begin{itemize}
            \item interaction patterns (request-response, publish-subscribe, etc.)
        \end{itemize}

        \bigskip

        \item[$\rightarrow$] basic communication primitives are not sufficient at this level
        %
        \begin{itemize}
            \item we need higher level notions such as \alert{interaction protocols}
            \item architectural entities may serve this purpose, e.g. brokers, servers, etc.
        \end{itemize}
    \end{itemize}

    \framebreak

    \begin{block}{\textbf{Engineers} must tackle \textbf{technical} aspects}
        \begin{enumerate}
            \item let several loci of control \alert{co-exist} in an efficient \& scalable way\ldots
            %
            \begin{itemize}
                \item possibly in spite of \alert{distribution}
                \item possibly in spite of different \alert{execution platforms}
            \end{itemize}
            \item make their interaction \alert{observable \& debuggable}
            \item chose the most adequate \alert{technological protocol}\ldots
            \item \ldots and data \alert{representation format}
            %
            \begin{itemize}
                \item in order to keep the system open, interoperable, evolvable, yet efficient
            \end{itemize}
        \end{enumerate}
    \end{block}

    \medskip

    \begin{itemize}
        \item engineers must thus be able to govern
        %
        \begin{itemize}
            \item middleware technologies (e.g. \jade{})
            \item architectural styles (e.g. ReST)
            \item application-level protocols (e.g. HTTP),
            \item data-representation formats (e.g. JSON, YAML)
        \end{itemize}
    \end{itemize}

    \framebreak

    \begin{block}{\textbf{System Engineers} must tackle \textbf{deployment}-related aspects}
        \begin{itemize}
            \item start system components in an \alert{orderly \& reproducible} way
            \item remove/replace/update/replicate system components \alert{gracefully}
            \item \alert{monitoring} and managing remote components
            \item ensure system components work \alert{predictably} despite heterogeneity
            \item support \alert{stability} despite the pace of software \alert{evolution}
        \end{itemize}
    \end{block}

    \begin{itemize}
        \item system engineers must thus be able to \emph{also} govern
        %
        \begin{itemize}
            \item containerisation and orchestration technologies (e.g. Docker)
            \item build automation technologies (e.g. Gradle, Maven, NPM, PIP)
            \item distributed version control systems (e.g. Git, Mercurial)
            \item package managers and repositories (e.g. Maven, NPM, Nuget, PyPi)
            \item multiple platforms and SDK (e.g. JVM, NodeJS, .NET, Python)
        \end{itemize}
    \end{itemize}

\end{frame}

\begin{frame}[c, allowframebreaks]{About the the Lab}

    \begin{block}{Rationale of the lab: make you \textbf{autonomous} in learning}
        \begin{itemize}
            \item Mastering distributed systems requires acquiring several notions
            \begin{itemize}
                \item[ie] being a modeller, designer, and (system) engineer, simultaneously
                \item[!] too many knowledge for the 6 ECTS credits of the course
            \end{itemize}

            \item so, we select a small core of fundamental notions to be acquired
            \begin{itemize}
                \item enabling your autonomy in learning advanced aspects
                \begin{itemize}
                    \item[eg] in later courses or as part of your future career
                \end{itemize}
            \end{itemize}
        \end{itemize}
    \end{block}

    \begin{block}{Goal of the lab: prevent \textbf{obsolescence} of skills}
        \begin{itemize}
            \item Mastering distributed systems practising with several technologies
            \begin{itemize}
                \item programming platforms / languages, protocols, tools, etc.
                \item[!] any technological choice will soon become \alert{outdated}
            \end{itemize}

            \item so, we select technologies which better exemplify relevant notions
            \begin{itemize}
                \item not necessarily the most efficient/updated ones
            \end{itemize}

            \item[!] we care you acquire durable \& general skill
            \begin{itemize}
                \item[$\rightarrow$] learning technologies is useful for today
                \item[$\rightarrow$] acquiring the underlying notions is fundamental for tomorrow
            \end{itemize}
        \end{itemize}
    \end{block}

    \begin{block}{Our approach: \textbf{bottom-up}}
        Make you
        %
        \begin{enumerate}
            \item confident with \alert{development tools}
            \item practice with basic aspects of \alert{remote communication}
            \item experiment with articulated \alert{interaction patterns}
            \item taste the benefits of leveraging on full-fledged \alert{middlewares}
        \end{enumerate}
    \end{block}
\end{frame}

\section{Organisation}

\subsection{Structure of Lab lectures}

\begin{frame}[c,allowframebreaks]{Structure of the Lab}

    An incremental path providing a taste of all the aforementioned issues: %(showing how main notions are built \& exploited C\&DS)
    %
    \medskip
    %
    \begin{enumerate}
        \item we will first present the fundamentals of
        %
        \begin{itemize}
            \item \alert{build automation} and \emph{dependency management} with Gradle
            \item \alert{containerisation} and \alert{orchestration} with Docker
        \end{itemize}

        \medskip

        \item we will then recall the basics of
        %
        \begin{itemize}
            \item \alert{multi-threading} and \alert{asynchronous programming}, on the JVM
            \item (byte-)stream-oriented \alert{communication via Sockets}, on the JVM
        \end{itemize}

        \medskip

        \item we will then present the notion of \alert{(de-)serialization}
        %
        \begin{itemize}
            \item and its usage for (un)marshalling data in remote communications
            \item proposing practical exercises with the Jackson data-processing library
        \end{itemize}

        \framebreak

        \item we will then present the notion of \alert{ReST API}
        %
        \begin{itemize}
            \item showing how they can be \alert{formalised} via Swagger
            \item discussing how \alert{web servers} can be realised on the JVM, via Javalin
            \item discussing how \alert{web clients} can be realised on the JVM
        \end{itemize}

        \medskip

        \item we will then describe state-of-the-art tools for
        %
        \begin{itemize}
            \item building \alert{service-oriented applications}, e.g. via gRPC and Protobuff
            \item \alert{temporally uncoupling} interacting entities, e.g. via RabbitMQ
            \item \alert{replicating} data over a number of distributed machines, e.g. via etcd
        \end{itemize}

        \medskip

        \item finally, we will exemplify notable \alert{agent-oriented infrastructures}
        %
        \begin{itemize}
            \item such as \jade{}\ccite{jadebook-2007}, TuCSoN\ccite{tucson-jir98}, or TuSoW\ccite{tusow-icccn2019}
        \end{itemize}

    \end{enumerate}

\end{frame}

\subsection{Required Technologies and Skills}

\begin{frame}[c,allowframebreaks]{Required Technologies and Skills}

    \begin{block}{Legend}
        \begin{multicols}{2}
            \begin{itemize}
                \item[$\checkmark$] we assume you know this topic
                \item[$\rightarrow$] we teach this topic
            \end{itemize}
        \end{multicols}
    \end{block}

    \framebreak

    \begin{alertblock}{Required}
        \begin{itemize}
            \item[$\checkmark$] basic understanding of computer networks
            %
            \begin{itemize}
                \item ISO/OSI and TCP/IP stacks
            \end{itemize}

            \item[$\checkmark$] distributed version control systems (DVCS) \& \alert{Git}
            %
            \begin{itemize}
                \item useful resources: \cite{pianiniDvcs, proGit}
            \end{itemize}

            \vfill

            \item[$\rightarrow$] build automation tools and \alert{Gradle}
            %
            \begin{itemize}
                \item useful resources: \cite{pianiniBuildAutomation, gradleUserGuide}
            \end{itemize}

            \vfill

            \item[$\rightarrow$] containerisation, orchestration and \alert{Docker}

            \vfill

            \item OO programming in Java, and, in particular:
            %
            \begin{itemize}
                \item[$\checkmark$] input/output API (useful resources: \cite{ProgrammizJavaIO})
                \item[$\checkmark$] collections API (useful resources: \cite{Naftalin2006, Bloch2008, JavaCollectionsCheatsheets})
                \item[$\checkmark$] streams and lambdas API (useful resources: \cite{Warburton2014, Bloch2008})
                \item[$\checkmark$] multi-threading API (useful resources: \cite{Lea1999, Oaks2004, Garg2004, Goetz2006})
                \item[$\rightarrow$] asynchronous programming API
            \end{itemize}

        \end{itemize}
    \end{alertblock}

    \begin{exampleblock}{Useful}
        \begin{itemize}
            \item[$\checkmark$] shell scripting and \alert{Bash}
            \item[$\checkmark$] basic IDE configuration and usage (\alert{Eclipse} or \alert{IntelliJ Idea})
        \end{itemize}
    \end{exampleblock}

    \begin{alertblock}{Setting up your own environment}
        \begin{itemize}
            \item[!] follow the instructions provided here: \cite{envSetup}
        \end{itemize}
    \end{alertblock}

\end{frame}

\section{Conventions and Suggestions}

\begin{frame}[c,allowframebreaks]{About Lab Exercises}

    \begin{itemize}

        \item lab lectures may involve one or more exercises
        %
        \begin{itemize}
            \item commonly started during the lab lecture
            \item rarely completed within the same lecture
        \end{itemize}

        \bigskip

        \item you are supposed to learn something, not just solving puzzles
        %
        \begin{itemize}
            \item no deadline, no mark
            \item[$\rightarrow$] take your time
            \item[$\rightarrow$] ask for help on the forum
            \item[$\rightarrow$] compare/discuss with your colleagues
        \end{itemize}

        \bigskip

        \item exercises are \alert{optional by default}
        %
        \begin{itemize}
            \item except those \emph{explicitly} marked as \alert{mandatory}
            %
            \begin{itemize}
                \item[$\rightarrow$] mandatory = ``check for comprehension will eventually be checked''
            \end{itemize}
        \end{itemize}

        \framebreak

        \item comprehension of Lab-related topics is checked before final exam:
        %
        \begin{itemize}
            \item \alert{expected workflow:} Lab activity check $\rightarrow$ Admission to final exam
            \item check performed by tutor(s)
            \item students must explicitly ask for the check, via email
        \end{itemize}

        \bigskip

        \item exercises are provided as GitLab repositories hosting Gradle projects
        %
        \begin{itemize}
            \item you need a GitLab account $\rightarrow$ follow the instructions described in \cite{envSetup}
        \end{itemize}

        \bigskip

        \item slides are provided as \emph{versioned} PDF, through GitHub releases
        %
        \begin{itemize}
            \item[eg] \uurl{\githubsd{lab-environment-configuration/releases}}
            \item[eg] \uurl{\githubsd{lab-about/releases}}
        \end{itemize}

    \end{itemize}

\end{frame}

\begin{frame}[c]{About Exercises Submissions}

    \begin{itemize}
        \item solutions must be pushed on the same GitLab repository exercises have been cloned from, \alert{through Git}

        \vfill

        \item to this end, you create a GitLab account, following instructions provided in \cite{envSetup}:
        %
        \begin{itemize}
            \item possibly, using your institutional credentials \texttt{\alert{name.surnameN}@studio.unibo.it}
            \item possibly, using \texttt{\alert{name.surnameN}} as username
        \end{itemize}

        \vfill

        \item after that, you are supposed to request access on the DS \academicYearShort{} GitLab group:
        %
        \begin{itemize}
            \item \url{\gitlabGroup}
        \end{itemize}

        \vfill

        \item solutions must be pushed on a branch named \alert{\texttt{submissions/\textit{name.surnameN}}}, as described in \cite{envSetup}
        %
        \begin{itemize}
            \item one solution per student ($\implies$ no group submissions)
            \item solutions provided elsewhere will be ignored
        \end{itemize}
    \end{itemize}
\end{frame}

\begin{frame}[c]{About Fora}
    \begin{itemize}
        \item use the general forum as much as possible
        \\
        \uurl{\generalForum}
        %
        \begin{itemize}
            \item don't be shy :)
            \item ask for help if you need it
            \item ask why if you are curious
            \item compare your solutions
            \item be critical and provide suggestions if you feel so
        \end{itemize}
    \end{itemize}
\end{frame}

\section{The Final Project}

\begin{frame}[c]{About Projects}
    \begin{itemize}
        \item detailed rules here
        \\
        \uurl{\projectRules}

        \vfill

        \item workflow overview
        %
        \begin{enumerate}
            \item \alert{choose} a project or \alert{propose} one
            \item reserve your project on the \href{https://virtuale.unibo.it/mod/forum/view.php?id=611834}{Projects forum}
            \item submit an \alert{initial report}, describing your own requirements
            \item receive a Git repository for tracking the development of your artefacts
            \item develop your projects
            \item write the \alert{final report}
            \item submit your project \alert{code \& report} and ask for lab activity check
            \item set up an appointment for discussing your project
        \end{enumerate}

        \vfill

        \item group projects are allowed (max 4 persons)
        %
        \begin{itemize}
            \item rule of thumb: $\sim90$ working hours per person per project
        \end{itemize}
    \end{itemize}
\end{frame}

\begin{frame}[c, allowframebreaks]{Sorts of Projects}

    \begin{block}{Classic DS Project}
        Develop a distributed application
        %
        \begin{itemize}
            \item[eg] web application, distributed video game, etc
            \item Workflow:
            %
            \begin{enumerate}
                \item Sketch the idea
                \item Design
                \item Write tests
                \item Implement
                \item Pack / deploy for reusability
            \end{enumerate}
            \item Previous projects in this category: \cite{Sd2021ProjectGuessR,Sd2021ProjectCAHu}
            \item[!] Innovation is limited here, thus we expect
            %
            \begin{itemize}
                \item design to be optimal, implementation to be complete
                \item all aspects of software engineering
                %
                \begin{itemize}
                    \item[ie] testing, packaging, continuos integration
                \end{itemize}
            \end{itemize}
        \end{itemize}
    \end{block}

    \begin{block}{Advanced DS Project}
        Provide an implementation for some distributed algorithm / protocol / architecture / model from the literature
        %
        \begin{itemize}
            \item[eg] a consensus protocol, a distributed store, a data sharing protocol, etc.
            \item Workflow:
            %
            \begin{enumerate}
                \item Select \& study a paper form the literature
                \item Design your implementation for the proposed contribution
                \item Prototype an implementation
                \item Pack / deploy for a demo
            \end{enumerate}
            \item Previous projects in this category: \cite{Sd2021ProjectZyzzyva, Sd2021HybridQuorum}
            \item[!] Innovation is high here, thus we
            %
            \begin{itemize}
                \item mostly care about the proposed design
                \item implementation can be sub-optimal or embryonic
                \item other aspects of software engineering have lower priority
            \end{itemize}
        \end{itemize}
    \end{block}

    \begin{block}{Research DS Project}
        Help us extending some research product of ours with some feature
        %
        \begin{itemize}
            \item[eg] TuCSoN\ccite{tusow-icccn2019}, TuSoW\ccite{tusow-icccn2019}, LPaaS\ccite{lpaas-bdcc2}, tuProlog\ccite{2pkt-softwarex}, etc.
            \item Workflow:
            %
            \begin{enumerate}
                \item Study the target research product
                \item Collect requirements by interacting with us
                \item Co-design the feature by interacting with us
                \item Implement \& test the feature
            \end{enumerate}
            \item Previous projects in this category: \cite{Sd2021ProjectTusowCSharpInterface}
            \item[!] Innovation is high here, but the activity is very constrained
            %
            \begin{itemize}
                \item design and implementation must be expert-level
                \item testing, and continuos integration are mandatory
                \item you will be considered part of development team
                \item you will be part of the author list of any subsequent paper
            \end{itemize}
        \end{itemize}
    \end{block}

    \begin{block}{Technology Know-How DS Project}
        Study a technology of choice in depth, producing a technical report
        %
        \begin{itemize}
            \item Workflow:
            %
            \begin{enumerate}
                \item Select a technology, study its documentation
                \item Understand its major abstraction
                \item Experiment with its deployment, and ordinary usage
                \item Stress it with extra-ordinary usage
                \item Report about the technology and the experiment
            \end{enumerate}
            \item Previous projects in this category: \cite{Sd2021ProjectEtcd,Sd2021ProjectCEPH,Sd2021ProjectKubernetes}
            \item[!] Focus here is on the technical report, other than the experiments
            %
            \begin{itemize}
                \item software artefacts here consist of the code of the experiments
                \item avoid copy-pasting the documentation, use your own words
                \item experiment should cover more than the official ones
            \end{itemize}
        \end{itemize}
    \end{block}
\end{frame}

%===============================================================================
\section*{}
%===============================================================================
\frame{\titlepage}

%===============================================================================
\section*{\bibname}
%===============================================================================

\setbeamertemplate{page number in head/foot}{}
%\\\\\\\\\\\\\\\\\\\\\
\begin{frame}[t,allowframebreaks,noframenumbering]\frametitle{\refname}
    % \begin{frame}[c]\frametitle{\refname}
    %    \footnotesize
    %    \scriptsize
        \tiny
    \bibliographystyle{plain}
    \bibliography{sd-lab-about}
\end{frame}
%\\\\\\\\\\\\\\\\\\\\\

%%%%%%%%%%%%%%%%%%%%%%%%%%%%%%%%%%%%%%%%%%%%%%%%%%%%%%%%%%%%%%%%%%%%%%%%%%%%%%%
\end{document}
%%%%%%%%%%%%%%%%%%%%%%%%%%%%%%%%%%%%%%%%%%%%%%%%%%%%%%%%%%%%%%%%%%%%%%%%%%%%%%%%
